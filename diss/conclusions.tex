\chapter{Conclusions}

This project was undertaken because the general standards for storing email have not changed substantially (apart from some performance improvements) for some time. This could be seen to be because they need no further improvement. However, I decided that this was not the case as the two main competing types of standard have different benefits that are not completely mutually exclusive. Therefore I set out to build a library that provided both elements of concurrency and strong consistency.

The library was built in OCaml using the structure of a Maildir inside a Git store. This required a large reimplementation of some internal Git operations to allow the necessary actions to be taken. It also involved having to think very carefully about how to build in good performance under concurrency when the action of committing to a Git store is a fundamentally sequential operation. This relied on seeing which operations were idempotent and using retries and rollbacks in the case of clashes but without re-running any idempotent actions. There was also the problem of having a daemon which could work independently of the command-line client whilst still being able to recognise the correct action to apply. This was achieved by using the existence of the Git history alongside diffing operations to infer the action that was taken.

In evaluating the project, I managed to show that although Maildir always outperformed Gitmaildir, Gitmaildir did outperform Mbox on more concurrent workloads in highly parallel environments. Unexpectedly, Gitmaildir outperformed Mbox significantly in time as well as speedup for email modification and had a more consistent speedup for that same operation compared to Maildir. These, together with the expected results, show that Gitmaildir is successful in providing a mailbox format that is consistent, performant, and can benefit from parallelism. Furthermore, the plugin system both works and does not introduce significant overhead meaning that Gitmaildir can outperform Mbox while simultaneously supporting more features.

\section{Further work}

The building of Gitmaildir creates the opportunity to explore this area further. The plugin system in conjunction with access to a Git repository means that novel features could be added very quickly to expand how we use our mailboxes. One such suggestion I had was to support true email threads where we can store separate threads in different branches to get a better overview of long-running conversations.

When it comes to the library itself, the next steps would be to implement Gitmaildir as part of an MTA (mail transfer agent) and MUA (mail user agent) as these would allow it to be used in a real environment without it relying on the daemon. Also, I believe further testing and a deeper exploration into mapping a Maildir to a Git repository could be fruitful in decreasing bottlenecks and aiding performance.

