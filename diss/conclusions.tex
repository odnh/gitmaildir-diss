\chapter{Conclusions}

This project was undertaken because the general standards for storing email have not changed substantially for a long time. This could be seen to be because they need no further improvement, however I decided that this was not the case as the two main competing standards both have different benefits that are not completely mutually exclusive. Therefore I set out to build a library that provided both elements of concurrency and strong consistency.

The library was implemented in OCaml using a structure of Git over a Maildir. Working with OCaml for this proved difficult at first on multiple fronts. These were that it was hard to work out a good way to structure the different parts of the library to allow enough separation to make it easily comprehensible while simultaneously having the code cohesive enough to allow a large deal of flexibility. It also proved challenging to implement some of the required algorithms especially while keeping track of all of the possible errors tidily. Coming to a decision of how to implement the plugin system had similar problems. However, all of the original success criteria were met and two extensions were completed successfully.

In evaluating the project I managed to show that although Maildir always outperformed Gitmaildir, Gitmaildir did outperform Mbox on more concurrent workloads in highly parallel environments. This was a good outcome as it was the intent behind the project to find a middle ground between these two standards. A middle ground where some of the speed and concurrency of Maildir is sacrificed to allow us to have a strongly consistent index. I also showed that the extra overhead introduced by using a plugin model to benefit from the extra features of Git was not significant beyond the preexisting overhead of Gitmaildir. This means that Gitmaildir can still outperform Mbox and support more features simultaneously.

\section{Further work}

The building of Gitmaildir creates the opportunity to explore this area further. The plugin system offered in conjunction to access to a Git repository means that novel features could be added very quickly to expand the ways in which we use our mailboxes. On such suggestion I had was to support true email threads where we can store separate threads in different branches to get a better overview of long-running conversations.

When it comes to the library itself, the next steps would be to implement Gitmaildir as part of an MTA (mail transfer agent) and MUA (mail user agent) as these would allow it to be used in areal environment without it relaying on the daemon. Also, I believe further testing and a deeper exploration int ways to map a Maildir to a Git repository could be fruitful in decreasing the bottlenecks and aiding performance further.

