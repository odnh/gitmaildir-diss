\chapter{Introduction}

\section{Motivation}

Email is a system of electronic communication supported by most personal computing platforms. It has been in use since the very beginnings of the Internet\footnote{This can be seen from an RFC for mailbox protocols being published in 1971\cite{rfc196}}. An essential part of this system is the ability to store messages. Storage is needed both for end users for later reading and on servers while emails await delivery. Over time there have been multiple different formats for email storage in mailboxes, one of the earliest and most widely used standards being Mbox. Another standard that was created later to deal with some of the shortcomings of Mbox was Maildir.

The fundamental difference between Mbox and Maildir is that Mbox relies on the entire mailbox being stored in a single file whereas Maildir keeps each email as a separate file in a directory. This makes Mbox very simple and places a strict ordering on the emails stored but does mean that only one process can read or write at once. Maildir has concurrency for reads and writes but loses the ordering that Mbox provided. This project seeks to find a middle ground where we can maintain some of the benefits that Maildir has such as concurrency whilst not losing the ability to have a consistent view of stored emails which Mbox offered.

\section{Related work}

There are many different implementations of mailboxes which all offer subtle improvements to Mbox and Maildir. They tend to follow either Mbox in having one file for the entire mailbox or Maildir in having separate files per email. However, they differ in how they format the storage of metadata. For example, dbox from Dovecot can be used in either configuration but holds an index of metadata separately. This allows for a lot of storage flexibility, but all operations require a complete lock over the mailbox for the entirety of the operation. mbx from UW-IMAP implements Mbox but with a fixed header size and more intelligent locking allowing concurrent reads and faster metadata modification, but it still lacks concurrency for other operations.

\section{Project goals}

It would be useful to have an implementation of a mailbox format which is both concurrent and scalable whilst also having a consistent view of the stored emails.

The project aims to solve the above problem by combining a Git (a distributed version control system introduced in more detail in section \ref{section:git}) layer with the Maildir format to provide strong consistency. Furthermore, the intent is that the use of Git will allow for the implementation of extra features that let us use and interact with a mailbox in new ways. This is like many of the improved mailbox formats mentioned above, except most focus on treating metadata in new ways at the cost of concurrency whereas Gitmaildir will focus on achieving consistency while maintaining concurrency.

The project will be achieved by creating an OCaml library providing the functionality, a command-line tool to interact with the mailbox and a daemon process to synchronise a standard Maildir with a Gitmaildir for easier interoperability. Also to be executed as extensions are a plugin system and a single demonstrative plugin for that system.

\section{Summary of work}

All the work specified was completed successfully: a library, client and daemon were all built and tested to function correctly. During the performance testing, it was shown that as expected, the new Gitmaildir underperformed in comparison to Maildir but outperformed Mbox entirely under some tests even when running in a single thread. This demonstrates that the tool and library support the performance necessary to be usable as an alternative to the current offerings. Furthermore, we have a complete history of all previous states of the mailbox and the ability to use this along with branching to implement plugins. Together, these offer quite a different set of features to all the other options at the current time while still being simultaneously faster under some conditions and offering the consistency that many of the faster solutions do not have.

