% arara: lualatex
\documentclass[12pt,a4paper,twoside]{article}
\usepackage[pdfborder={0 0 0}]{hyperref}
\usepackage[margin=25mm]{geometry}
\usepackage{graphicx}
\usepackage{parskip}

\begin{document}

\begin{center}
    \Large
    Computer Science Tripos -- Part II -- Progress Report\\[4mm]
    \LARGE
    A strongly consistent index for email using git and MirageOS.

    \large
    O.~Hope, Jesus College

    Originator: Dr A.~Madhavapeddy

    28th January 2019
\end{center}

\vspace{5mm}

\textbf{Project Supervisor:} Mr D.~Allsopp

\textbf{Director of Studies:} Prof C.~Mascolo

\textbf{Project Overseers:} Prof G.~Winskel \& Dr R.~Mortier

% Main document

\section*{Progress compared to the schedule}

Currently the project is on track compared to the original schedule. According to the proposal, at this point the milestone that was supposed to be reached was `OCaml application complete' and this is the case. Furthermore, the next milestone is `Testing of all basic features complete' and I have begun building test frameworks to gather data for evaluation alongside starting on some parts of the extensions. This shows that while the project is not ahead of schedule, it is also not behind, so should be finished in good time.

\section*{Difficulties encountered}

The project has run relatively smoothly overall. However, I did have to modify the schedule slightly, moving from prototyping to implementation more quickly than originally expected. This was because there was more that needed re-implementing in OCaml compared to the commandline tool than I had at first anticipated. Also, I had some difficulty at first working out how to use the necessary libraries due to a lack of complete doeumentation. However, at this point all of these problems have been resolved.

\section*{What has been accomplished}

Thus far I have created a library in OCaml which, with the help of a MirageOS library called ocaml-git, allows me to perform git plumbing commands in OCaml. Using this I have then implemented the necessary functions for basic interaction with a maildir in a git store through OCaml. Finally, I have created a commandline interface that wraps this, allowing for interaction from users and other programs. This tool also syncs a standard maildir with the git store and handles any issues of concurrency upon modification attempts from multiple concurrent processes.

\end{document}
