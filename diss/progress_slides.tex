% arara: lualatex: { synctex: yes, shell: yes }
\documentclass{beamer}
\usetheme{metropolis}

% Uncomment for handout
%\usepackage{pgfpages}
%\pgfpagesuselayout{4 on 1}[a4paper,border shrink=5mm,landscape]

\title{A strongly consistent index for email using git and MirageOS}
\date{}
\author{Oliver Hope}
\institute{Jesus College}
\begin{document}

  \maketitle

  \begin{frame}[fragile]{Background}
    UNIX Mail storage

    \begin{itemize}
      \item mbox
      \item maildir
    \end{itemize}

    What is Git?

    \begin{itemize}
      \item Content-Addressable Storage
      \item Porcelain
      \item Plumbing
    \end{itemize}
  \end{frame}

  \begin{frame}[fragile]{What has been built?}
    What has been built?

    \begin{itemize}
      \item \emph{gitmaildir} --- A library
      \item \emph{gitmaildir\_cli} --- A command line client
      \item \emph{gitmaildir\_daemon} --- A daemon
    \end{itemize}
  \end{frame}

  \begin{frame}{How has it been built?}
    Everything built in OCaml. This is because MirageOS has some very useful libraries for what is being done:

    \begin{itemize}
      \item ocaml-git
      \item MrMime
    \end{itemize}

    Uses OCaml functors --- Easy to swap out backend

    Written using a concurrency monad (Lwt) --- Easy scaling
  \end{frame}

  \begin{frame}{What's left?}
    The core features are all complete, now time for testing:
    \begin{itemize}
      \item Timings of different operations
      \item Timings of different concurrency implementations
    \end{itemize}

    And time for extensions:
    \begin{itemize}
      \item Separate branches for mobile clients that strip data-heavy content and attachments from emails
      \item Branches per reply thread for a better insight and cleaner overview of long-running email threads
      \item Intelligent rollback in case of accidental deletion or modification
    \end{itemize}
  \end{frame}
\end{document}
