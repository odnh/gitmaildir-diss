\chapter{Preparation}

In this chapter I introduce the changes that I made to my original plan and the more detailed requirements that I decided my implementation should fulfill along with the reasons for doing so. Furthermore, I introduce the elements and workings of Git that were needed to be taken into account as part of the design and the intricacies of programming in OCaml that influenced the structure of the project in a different way to how it may have been written using an imperative or object oriented programming language such as C or Java.

\section{Plan Refinement}

\subsection{Requirements}

\subsection{Plan modifications and clarifications}

\section{OCaml}

The language of choice for this project was OCaml \cite{MinskyYaron2013RwO}. This is due to it being the language that MirageOS is implemented in and so there were preexisting libraries that were useful to use as part of the project. OCaml is a programming language which supports imperative, functional, and OOP programming styles. However, 

\subsection{Functional programming}

\subsection{Functors}

\subsection{Monads}

\section{Git}

\subsection{Content Addressable Storage}

\subsection{Porcelain and Plumbing}

\section{ocaml-git}

Ocaml-git \cite{ocaml-git} is a library that the project relies on heavily.

\section{Maildir}
