% arara: latexmk: { engine: lualatex, options: [-shell-escape, -bibtex, -synctex=1] }
\documentclass[12pt,a4paper,twoside,openright]{report}
\usepackage[pdfborder={0 0 0}]{hyperref}    % turns references into hyperlinks
\usepackage[margin=25mm]{geometry}  % adjusts page layout
\usepackage{graphicx}  % allows inclusion of PDF, PNG and JPG images
\usepackage{verbatim}
\usepackage{docmute}   % only needed to allow inclusion of proposal.tex
\usepackage{makecell}  % for overfull title table cell in proforma
\usepackage{adjustbox} % for the proforma table (resizes to get rid of latex warnings)

\raggedbottom                           % try to avoid widows and orphans
\sloppy
\clubpenalty1000%
\widowpenalty1000%

\renewcommand{\baselinestretch}{1.1}    % adjust line spacing to make
                                        % more readable

\begin{document}

\bibliographystyle{plain}


%%%%%%%%%%%%%%%%%%%%%%%%%%%%%%%%%%%%%%%%%%%%%%%%%%%%%%%%%%%%%%%%%%%%%%%%
% Title


\pagestyle{empty}

\rightline{\LARGE \textbf{Oliver Hope}}

\vspace*{60mm}
\begin{center}
\Huge
\textbf{A strongly consistent index for email using git and MirageOS.} \\[5mm]
Computer Science Tripos -- Part II \\[5mm]
Jesus College \\[5mm]
\today  % today's date
\end{center}

%%%%%%%%%%%%%%%%%%%%%%%%%%%%%%%%%%%%%%%%%%%%%%%%%%%%%%%%%%%%%%%%%%%%%%%%%%%%%%
% Proforma, table of contents and list of figures

\pagestyle{plain}

\chapter*{Proforma}

\begin{adjustbox}{width={\textwidth},totalheight={\textheight},keepaspectratio}%
{\large
\begin{tabular}{ll}
Name:               & \bf Oliver Hope \\
College:            & \bf Jesus College \\
Project Title:      & \makecell[l]{\bf A strongly consistent index for email using git \\ \bf and MirageOS.} \\
Examination:        & \bf Computer Science Tripos -- Part II, June 2019 \\
Word Count:         & \bf \input{|"texcount -sum -1 diss.tex"}\footnotemark[1] \\
Line Count:         & INSERT LINE COUNT \\
Project Originator: & Dr M.~Madhavapeddy \\
Supervisor:         & Mr D.~Allsopp \\
\end{tabular}
}
\end{adjustbox}
\footnotetext[1]{This word count was computed
by \texttt{texcount -sum -1 diss.tex}
}
\stepcounter{footnote}


\section*{Original Aims of the Project}

The original aims of the project were to create a new type of maildir\cite{maildir-man} on top of a git store. This was to provide the consistency that maildir does not afford whilst mitigating some of the problems with mbox that inspired the creation of maildir in the first place. Futhermore using git for mail storage allows for many novel features that can be implemented, and some examples of such features were a further aim.

\section*{Work Completed}

The core of the project built was a library, gitmaildir, which handles interaction with a gitmaildir. Then there are also two helper tools: gitmaildir\_cli a command-line interface for gitmaildirs and gitmaildir\_daemon, a daemon to sync a gitmaildir with a standard maildir. Also two extensions have been completed: branching based on reply threads and branching for preprocessing of mail for different devices. Of course as part of this it was necessary to design how a gitmaildir would work and write some code to create metrics for performance testing.

\section*{Special Difficulties}

There have been no special difficulties.

\newpage
\section*{Declaration}

I, Oliver Hope of Jesus College, being a candidate for Part II of the Computer
Science Tripos, hereby declare
that this dissertation and the work described in it are my own work,
unaided except as may be specified below, and that the dissertation
does not contain material that has already been used to any substantial
extent for a comparable purpose.

\bigskip
\leftline{Signed}

\medskip
\leftline{Date}

\tableofcontents

\listoffigures

\newpage
%%%%%%%%%%%%%%%%%%%%%%%%%%%%%%%%%%%%%%%%%%%%%%%%%%%%%%%%%%%%%%%%%%%%%%%
% now for the chapters

\pagestyle{headings}

\chapter{Introduction}

\section{Motivation}

Email is a system of electronic communication supported by most personal computing platforms. It has been in use since the very beginnings of the internet. An important part of this system is the ability to store messages. This can be both for end users for later reading and on servers while awaiting delivery. Over time there have been multiple different formats for email storage in mailboxes, one of the earliest and most widely used standards being mbox.

\subsection{Mbox}

Mbox is in fact not one single mailbox format, but is instead a family of similar formats. What they all share in common is that they are based around the idea that an entire mailbox is stored in a single plaintext file which consists solely of a long list of emails and their content. The first version, created for 5th edition Unix, contained all the emails one after another delimited by the characters: \texttt{FROM:} at the start of a new line.

The downsides of such an approach stem from the use of a single file and plaintext email storage. Due to the use of a single file it is not possible to have concurrent access to an mbox. This is because reading whilst someone else writes could lead to a corrupted read, and writing whilst someone else writes could corrupt the mbox in memory. The effect of this is that access and modifications are not scalable to higher levels of concurrency because it is not possible to perform operations at the same time. The other impact is that a single very large file cannot easily be distributed over multiple storage media meaning that mboxes are hard to scale for very large mailboxes. A problem which impacts implementation is that it is very possible for \texttt{FROM:} to appear in a message. This would be classed as a new email if we wrote it directly so we have to escape this usage and then unescape it when reading back the email.

The mbox approach does have some benefits. Most notably, it is very simple to implement, and as we can guarantee that writes are sequential we will always have a consistent view of our mailbox.

\subsection{Maildir}

An alternative to mbox was created much later was Maildir, intended to solve many of the problems associated with using mbox. The architecture is entirely different, with each email being stored in a separate, uniquely named file. There are also specific rules about how files should be added to the store and modified once there.

Due to its design, Maildir solves many of the problems mbox suffers from mentioned above. Having separate files allows for total concurrency in reading and writing (as long as you are not trying to modify the same email) and scalability of size is not an issue as it is relatively easy to distribute the storage of separate files across different storage media. Another benefit is that the semantics of email storage are completely removed from their contents meaning issues such as the message delimiting one mbox suffers from cannot occur. Finally, the mail clients reading a Maildir do not actually have to worry about locking as this is all handled by the filesystem, simplifying the implementation of the clients.

Whilst there are many benefits to this approach, there are two downsides compared to mbox. The first is that it is more complicated, but more importantly we can no longer guarantee a consistent view of the mailbox.

\section{Project goals}

It would be useful to have an implementation of a mailbox format which had the benefits Maildir brings in both concurrency and scalability whilst not losing the ability to have a consistent view of the stored emails. The reason this strong consistency is beneficial is that ... !!!PUT REASONS HERE!!!.

The project aims to solve this problem by combining the Maildir format with a Git layer to provide strong consistency. Furthermore the intent is that it should function on MirageOS, allowing it to be used as a mail storage backend and the use of Git will allow for the implementation of extra features allowing for new ways to use and interact with a mailbox.

\chapter{Preparation}

\section{Plan Refinement}

\subsection{Requirements}

\subsection{Plan modifications and clarifications}

\section{OCaml}

\section{Git and CASs}

\chapter{Implementation}

\section{Bash prototype}

\section{Gitmaildir}

\subsection{Structure Overview}

\subsection{Library}

\subsection{Command line interface}

\subsection{Daemon}

\subsection{Build system}

\section{Extensions}

\section{Repository Overview}

\chapter{Evaluation}

\section{Tests}

\subsection{Performance under test}

\subsection{Metrics}

\section{Test suite}

\section{Success critera}

\subsection{Successful functionality}

\subsection{Strong consistency}

\chapter{Conclusions}

%%%%%%%%%%%%%%%%%%%%%%%%%%%%%%%%%%%%%%%%%%%%%%%%%%%%%%%%%%%%%%%%%%%%%
% the bibliography
\addcontentsline{toc}{chapter}{Bibliography}
\bibliography{refs}

%%%%%%%%%%%%%%%%%%%%%%%%%%%%%%%%%%%%%%%%%%%%%%%%%%%%%%%%%%%%%%%%%%%%%
% the appendices
\appendix

\chapter{Project Proposal}

% arara: lualatex
% Note: this file can be compiled on its own, but is also included by
% diss.tex (using the docmute.sty package to ignore the preamble)
\documentclass[12pt,a4paper,twoside]{article}
\usepackage[pdfborder={0 0 0}]{hyperref}
\usepackage[margin=25mm]{geometry}
\usepackage{graphicx}
\usepackage{parskip}

\begin{document}

\begin{center}
    \Large
    Computer Science Tripos -- Part II -- Project Proposal\\[4mm]
    \LARGE
    A strongly consistent index for email using git and MirageOS.

    \large
    O.~Hope, Jesus College

    Originator: Dr A.~Madhavapeddy

    18th October 2018
\end{center}

\vspace{5mm}

\textbf{Project Supervisor:} Mr D.~Allsopp

\textbf{Director of Studies:} Prof C.~Mascolo

\textbf{Project Overseers:} Prof G.~Winskel \& Dr R.~Mortier

% Main document

\section*{Introduction}

Maildir is a widely used format for storing emails. Its main benefit is that it uses the filesystem in such a way that client programs do not have to handle locking themselves. The downside of this is that it makes it hard to create a consistent index as we cannot guarantee that the filesystem is in a consistent state when we try to update it. If we did have a consistent index, it would allow for safer concurrent support and the implementation of new features.

MirageOS is a library operating system for building unikernels written in OCaml. This means that it provides OCaml libraries which can be combined and compiled down into high-performance, lightweight kernels. These can then run directly on a hypervisor and only contain the code required to provide the functionality needed. However, the libraries themselves can also be used outside of unikernels to provide functionality to applications on more standard systems. This will be the use case in this project, using the MirageOS libraries to create a standalone application to run on unix-like systems.

The aim therefore is to solve the consistency problem. This can be done by using git, the version control system, to build an overlay on top of maildir in the filesystem. Then multiple filesystem operations can be bundled into commits. These can be used to track of all changes made to the maildir. As these changes are being recorded by a version control system, we can be sure that any index built on top will be strongly consistent. As git also provides branching, we can extend this model to add new features described in the possible extensions section.

We can implement this git overlay using libraries provided by MirageOS which provide git functionality, maildir operations, and even email parsing. With the overlay, and therefore consistent index implemented, we will then be able to make many more guarantees about the state of the maildir at any time. This will also allow for dealing with conflicting operations in an easier and more reliable manner. Furthermore, the overlay will also provide the possibility of easily implementing novel features such as roll-back and separate branches for different use cases.

\section*{Starting point}

I have used git for version control of software projects in the past, so I have a basic understanding of some of its more commonly used features when using it as a tool. However, I have little knowledge of its more low level actions which will be needed when managing filesystem operations.

MirageOS provides some libraries meaning that I will not have to implement some features from scratch. These include ocaml-git to manage git operations, a library for actioning maildir operations, and one which supports email file parsing.

I have previous experience in Standard ML from the Foundations of Computer Science course in Part IA, but most of the differences and extra features in OCaml will be new to me.

\section*{Resources required}

For this project I will use my own Apple Macbook Pro (2018) running macOS.
I will backup all of my work on a Github repository, and will make weekly backups of my entire computer to an external hard drive. Should my laptop suddenly fail, I have a spare and the option to use the MCS computers in the Computer Laboratory. Alongside my computer, I will take a snapshot of my personal Cambridge email account at the start of the project for testing purposes all other tests will be performed with generated mailboxes. I will require no other special resources.

\section*{Work to be done}

The project breaks down into the following sub-projects:

\begin{enumerate}

    \item \textbf{Designing overlay functionality:} Working out exactly what git features will be needed for the required overlay functionality, how they fit together, and designing a strategy to implement this.

    \item \textbf{Prototyping the overlay:} Building a version of the overlay to work using standard command line tools (including git) and non-mirage scripts.

    \item \textbf{Adding index functionality:} Using the overlay to actually implement the prototype of the consistent indexing.

    \item \textbf{Adapting the overlay to MirageOS:} Implementing the overlay with MirageOS libraries and OCaml using the knowledge gained whilst prototyping, using pre-existing libraries to manage the git and filesystem operations.

    \item \textbf{Testing and Evaluation:} Checking that all maildir operations continue to work as expected, testing the index for strong consistency. Testing resolution of conflicts from concurrent operations. Quantifying the slow-down relative to maildir with no consistent index over operations such as insertion and deletion of emails in different quantities and concurrent actions from different processes.

\end{enumerate}

\section*{Success criteria}

The success of the project will be evaluated against:

\begin{enumerate}

    \item Implementation of a working git overlay on top of the filesystem in MirageOS.

    \item A strongly consistent maildir index using the git overlay which allows all standard maildir operations.

\end{enumerate}

\section*{Possible extensions}

If I finish both my implementation and evaluation of the project early, then it could be extended by implementing:

\begin{itemize}

    \item Rollback support in case of user errors such as accidental deletion.

    \item Branch support to process emails for different platforms. For example a branch for mobile clients which has all images removed from the emails to reduce mobile data usage.

    \item Intelligently handling conflicts such as replying to the same message twice from different clients.

\end{itemize}

\section*{Timetable}

Planned starting date is 22/10/2018.

\begin{enumerate}

    \item \textbf{Michaelmas weeks 2--4 (22nd October--31st October):} Read through the full git documentation of all the standard features, research the maildir format, evaluate different indexing techniques, start learning OCaml, and take a snapshot of my Cambridge email account.

        \emph{Milestone: Have decided on a high level implementation strategy}

    \item \textbf{Michaelmas weeks 5--6 (1st November--14th November):} Write the prototype overlay using git on a standard OS (macOS) and continue learning OCaml.

    \item \textbf{Michaelmas weeks 7--8 (15th November--28th November):} Add the indexing functionality, test under all the necessary operations, read the MirageOS documentation relating to the filesystem, git operations, and any other libraries that will be needed.

        \emph{Milestone: Prototype implementation complete}

    \item \textbf{Michaelmas vacation weeks 1--2 (29th November--12th December):} Start implementation of the overlay in OCaml.

    \item \textbf{Michaelmas vacation weeks 6--7 (3rd January--16th January):} Continue implementation of overlay in OCaml, and test index is consistent for all operations.

        \emph{Milestone: OCaml application complete}

    \item \textbf{Lent weeks 1--2 (17th January--30th January):} Write progress report. Start to set up further tests for the application and begin evaluating its features.

    \item \textbf{Lent weeks 3--4 (31st January--5th February):} Set up and perform tests to evaluate the performance of the overlay under different operations and maildir conditions.

        \emph{Milestone: Testing of all basic features complete}

    \item \textbf{Lent weeks 5--6 (14th February--27th February):} Implement possible extensions if all previous milestones met and begin writing main dissertation chapters.

    \item \textbf{Lent weeks 7--8 (28th February--13th March):} Complete dissertation first draft and complete extensions.

        \emph{Milestone: Dissertation draft complete}

    \item \textbf{Easter vacation (20th March--24th April):} Start revising dissertation using feedback given, and complete any remaining evaluations.

    \item \textbf{Easter term weeks 1--2 (25th April--8th May):} Complete dissertation final draft.

        \emph{Milestone: Dissertation complete}

    \item \textbf{Easter term week 3 (9th May--15th May):} Any remaining proofreading, and an early submission to concentrate on exam revision.

        \emph{Milestone: Dissertation submitted}

\end{enumerate}

\end{document}


\end{document}
